\chapter{Conclusions and Future Work}
\label{cha:conclusions}

This Master Thesis project has been divided in three major goals. The design of
ATD, the implementation of ATDlib, and the development of the algorithm behind
the Model Judge platform. We conclude that the three main initial goals have
been achieved successfully.

% ATD

% - Robust and understandable way to document business processes 
% - Language is still in its early stages. Needs to solidify, for that, we need to
% test it by annotating models.
% - There is a lack of training data for NLP in BPM. Mainstream adoption of ATD
% could change that.

\emph{ATD} are a novel alternative for business process documentation based on natural
language. Being a more understandable alternative to classical graphical and
rule-based notations, we expect ATD to be a useful documentation format for
organizations, as well as enable the same automation capabilities as more formal
languages, like BPMN. Finally, being annotated texts, mainstream adoption of ATD
can help fill the lack of training data in the field of Natural Language
Processing for Business Process Management. Future work in the development of
ATD will involve solidification of the language: Deploy ATD-based solutions in
different application domains to see how to adapt the base language to the wide
variety of requirements in organizations.

% ATDlib

% - Importance of a reference implementation in ATDlib.
% - For now it's just a prototype, we must look carefully how to integate into
% many application domains
% - Also provide functionality to work with ATD expansions as part of ATDlib.
% - Although the backend is implemented in Clojure, include a future Java API to
% aid in adoption in organizations using other JVM languages (Scala, Kotlin...).
% - Provide some kind of integration with annotation sofware. Brat is
% half abandoned, TAG looks promising

\emph{ATDlib} is the reference implementation of ATD. It can parse, automatically
annotate and convert ATD to various useful representations. The current
implementation is still in a very early stage, but it has been sucessfully
deployed in the Model Judge platform. However, further effort is necessary to
allow easy integration of ATDlib into any BPM system. One of the action points
in that direction is to provide a Java interface into the library to aid in ATD
adoption for organizations using any other JVM language. Another remaining
functionality is to add support for ATD extensions into the library. Finally,
to allow for easy generation and annotation of ATD, a visual front-end must be
developed by fully integrating ATDlib into a text annotation interface.

% Model Judge

% - ATD in a real use-case.
% - helps students in the classroom with a high degree of satisfaction.
% - Has been tested in two real courses. More to come.
% - The analytics information has provided valuable insights. Find a good metric
% to evaluate whether students consider the feedback and how that affects their performance
% - Think about Model Judge outside of teaching. Assisted modeling software.
% - More diagnostic types need to be added. Especially improve control flow detection

\emph{Model Judge} can helps students learn modeling notations, such as BPMN and
has already been tested in two pilot university courses. It is thus an example of
the usage of Annotated Textual Descriptions in a real use-case. Future
work directions with Model Judge will come in two ways. In one hand, we want to
expand the usage of \emph{Model Judge} to more courses and universities. This
will require considering new kinds of useful diagnostics, and especially
implementing a robust way for fine-grained detection of control flow
inconsistencies in process models. On the other hand, we want to get more
detailed insights into the students' \emph{process of process modeling}. By
collecting anonymised and consented analytics information with \emph{Model
  Judge}, we believe we can help get a better grasp of how people learn to model
information, which will in turn help us develop better and more intuitive
models.