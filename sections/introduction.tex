\chapter{Introduction}
\label{cha:intro}

\section{Motivation}

%A model is defined as an abstract representation of an idea. As humans, we build
%and share models in order to transmit our mental representation of reality and,
%as such, they are one of the main building blocks of our organizations. In the field of
%computer science, we encounter a plethora of modeling languages for different
%application domains, from the Unified Modelling Language (UML) to First Order
%Logic itself.

%When we reach consensus on a modeling language, we can employ its vocabulary in
%order to communicate ideas to others unambiguously, which is why models are so
%important in cooperative tasks. When the scale of cooperation is at the level of
%large organizations, having a common, unambiguous understanding of the situation
%becomes crucial for operations.\todo{The first two paragraphs do not add much
  %contribution to the paper. However I believe they may be useful to
  %contextualize the work at a very general level. Should I keep them?}

\textit{Business Process Management} (BPM) is an area in operations research
with the goal of modeling, analyzing and optimizing \textit{business processes}.
One of the main pillars of BPM are \emph{process models}. A process model 
represents how a certain task is performed in an organization, and as such, can be
used to efficiently communicate the internal operations of a company.

In the era of information systems and artificial intelligence, the
focus of process models has progressively shifted from \textit{human readability} to
\textit{machine readability} \cite{ter2009modern}. We have moved from the traditionally graphical
models for business process documentation, such as the flow chart, to more
formal standards built on top of machine-readable languages, like the Business
Process Model and Notation (BPMN) \cite{chinosi2012bpmn} standard. This has
enabled automation for the analysis and improvement of business processes, which
in turn translates in real value for companies using these techniques.

However, this formalization of the semantics of process models results in an
increase in modeling languages' complexity, which in turn, results in
non-experts --stakeholders and employees alike-- having difficulties in
understanding the processes \cite{leopold2014supporting}. These misunderstandings may
have serious consequences for any organization \cite{van2015fragmentation}

For the above reasons, no standard for process modeling has ultimately stood above
the others. This has lead to a scenario where BPM-focused organizations often
have to maintain multiple representations of the same process model. Informal
versions, which heavily rely on semi-structured or unstructured natural
language, are kept next to more formal and structured alternatives. Furthermore,
in this formal spectrum, we encounter several modeling languages, each one
aiming to solve a different subset of problems \cite{10.1007/978-3-540-72035-5_7,
 van2003workflow}.

In this work, we propose an alternative to conventional process modeling
languages that aims to be intuitive and ergonomic, but at the same time acts 
as a formal representation enabling the automation of analysis, monitoring, or
simulation of business processes. Our technique combines the flexibility of
unstructured human langauge with Natural Language Processing (NLP) techniques in
order to create an Annotated Textual Description (ATD) that represents the
business process. Additionally, we present a practical application of ATD with
\emph{Model Judge}, a web platform designed to aid novices in the creation of
business process models which uses ATD as one of the core ideas behind its
algorithm.

\section{Project Scope}
\label{sec:scope}

The work presented in this Master Thesis project has been divided into the
following activities:

\subsection*{Formalization of ATD and its semantics}

The goal of this task is the creation of a robust language with well defined
semantics that is able to represent business processes based on annotated 
natural language descriptions. 

As part of this task, we have explored the common characteristics in business
process descriptions, in order to identify the relevant types of information for
the documents in our problem domain.

Next, the identified concepts have been translated in terms of an annotation
language: Annotations, and the set of relations between them. 

%Finally, due to the close relation between text annotations and ontologies, we
%have studied the possibility of mapping the concepts in ATD to some well known
%process description ontologies, such as the Process Specification
%Language\cite{gruninger2003process}.

\subsection*{Development of \emph{ATDlib} and Integration with Annotation Tools}

This task consisted in the development of a hybrid Clojure/Java library, \emph{ATDlib}.
This library is intended to be a set of software tools to manage and generate
ATD process descriptions.

The two main functionalities implemented in \emph{ATDlib} are the automatic
generation of partial ATD from text and the translation between ATD and
FreeLing \cite{PadroS12} semantic graphs.

In order to operationalize the workflow presented in this work for the
generation of ATD, the functionalities presented in \emph{ATDlib} must be
integrated into a graphical annotation tool. As part of this task, we explored
two alternatives in the NLP field: Brat \cite{stenetorp2012brat} and TAG
 \cite{DBLP:journals/corr/abs-1711-00529}.

\subsection*{Design and Development of the \emph{Model Judge}'s Algorithm}

This activity consisted in the design of an algorithm behind \emph{Model Judge},
a web platform designed to aid novice students in the process of process
modeling. This algorithm is based on the notion of Annotated Textual
Descriptions that was formalized and developed in the two previous tasks.

\subsection*{Analysis of the \emph{Model Judge} Analytics Data}

In order to gain insights and evaluate the \emph{Model Judge} platform, we
collected analytics information from two courses that took place in
Denmark's Technical University (DTU) and the Universidad Catolica de Santa
Mar\'ia (UCSM), in Peru, which used \emph{Model Judge} as the main
modeling tool.

This task consisted in the preprocessing and analysis of the data collected from
the platform.

%\begin{itemize}
  %\item The formalization of the definition of ATD and its semantics.
  %\item The development of ATDlib, a set of software tools to manage and
    %generate ATD.
  %\item Evaluate the integration between ATDlib and standard text annotation
    %tools
  %\item The design and development of the diagnostic algorithm behind
    %\emph{Model Judge}
%\end{itemize}

%The ATD semantics presented in this work have undergone several revisions in
%order to improve the robustness and ergonomics of the language. We believe the
%final formalisation presented can serve as a strong foundation for future work
%in this area.

%With ATDlib, we have achieved two very important goals in the operationalization
%of ATD process descriptions: Automatic generation of partial annotations and
%Automatic conversion of annotations to FreeLing document descriptions.

%The Model Judge platform has received very positive feedback and the evaluations
%provided show 

\section{Structure of this Document}

The remainder of this document is structured as follows:
Chapter~\ref{cha:background}, discusses the state of the art in the several
fields involved in this project and introduces all the necessary background
concepts. Chapter~\ref{cha:atd} presents Annotated Textual Descriptions and
discusses some of its theoretical applications in the field of Business Process
Management. Chapter~\ref{cha:atdlib} presents the technical details behind the
implementation of \emph{ATDlib}. Chapter~\ref{cha:modeljudge} describes the
\emph{Model Judge} platform, a practical use case of ATD and offers an empirical
evaluation of the platform with two modeling courses. Finally,
Chapter~\ref{cha:conclusions} concludes this thesis and discusses some possible
future research lines.
