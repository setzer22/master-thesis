\chapter{Introduction}
\label{cha:intro}

A model is defined as an abstract representation of an idea. As humans, we build
and share models in order to transmit our mental representation of reality and,
as such, they are one of the main building blocks of our organizations. In the field of
computer science, we encounter a plethora of modeling languages for different
application domains, from the Unified Modelling Language (UML) to First Order
Logic itself.

When we reach consensus on a modeling language, we can employ its vocabulary in
order to communicate ideas to others unambiguously, which is why models are so
important in cooperative tasks. When the scale of cooperation is at the level of
large organizations, having a common, unambiguous understanding of the situation
becomes crucial for operations.\todo{The first two paragraphs do not add much
  contribution to the paper. However I believe they may be useful to
  contextualize the work at a very general level. Should I keep them?}

\textit{Business Process Management} (BPM) is an area in operations research
with the goal of modeling, analyzing and optimizing \textit{business processes}.
One of the main pillars of BPM are \emph{process models}. A process model 
represents how a certain task is performed in an organization, and thus, can be
used to efficiently communicate the internal operations of a company.

In the era of information systems and artificial intelligence, the
focus of process models has progressively shifted from human readability to
\textit{machine readability}. We have moved from the traditionally graphical
models for business process documentation, such as the flow chart, to more
formal standards built on top of machine-readable languages, like the Business
Process Model and Notation (BPMN) \cite{chinosi2012bpmn} standard. This has
enabled automation for the analysis and improvement of business processes, which
in turn translates in real value for companies using these techniques.

However, this formalization of the semantics of process models results in an
increase in modeling languages' complexity, which in turn, results in
non-experts --stakeholders and employees alike-- having difficulties in
understanding the processes\cite{citation needed}. This often leads to
misunderstandings that cause errors in the execution of processes.

For the above reasons, no standard for process modeling has ultimately stood above
the others. This has lead to a scenario where BPM-focused organizations often
have to maintain multiple representations of the same process model. Informal
versions, which heavily rely on semi-structured or unstructured natural
language, are kept next to more formal and structured alternatives. Furthermore,
in this formal spectrum, we encounter several modeling languages, each one
aiming to solve a different subset of problems\cite{workflow patterns}.

In this work, we propose an alternative to conventional process modeling
languages that aims to be intuitive and ergonomic, but at the same time can act
as a formal language enabling the automation of analysis, monitoring, or
simulation of business processes. Our technique combines the flexibility of
unstructured human langauge with Natural Language Processing (NLP) techniques in
order to create an Annotated Textual Description (ATD). Additionally, we present a
practical application of ATDs with Model Judge, an web platform designed to aid 
novices in the creation of business process models which uses ATDs as one of the
main elements of its algorithm.

The remainder of this document is structured as follows. In
Chapter~\ref{cha:background}, we discuss in detail the state of the art of the
several fields involved in this project and introduce all the necessary
background concepts. Chapter~\ref{cha:atd} presents Annotated Textual
Descriptions and discusses some theoretical applications in the field of
Business Process Management. Chapter~\ref{cha:modeljudge} describes the
practical case of the Model Judge and offers an empirical evaluation of the
platform with two real modeling courses. Finally, Chapter~\ref{cha:conclusions}
concludes this thesis and discusses some possible future research lines.







%\begin{itemize}
%\item Problem description
%\item Purpose of the study
%\item Present the practical case
%\item Summarize results
%\end{itemize}