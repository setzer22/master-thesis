\chapter{Describing Processes with Natural Language}
\label{cha:atd}

\section{Annotated Textual Descriptions}

- Present ATDs as a PM notation
- Advantages
  - Can model ambiguity (flexible)
  - Open-world assumption: What's not explicit could be anything
  - Little representational bias 

- Requirements
  - Must be able to represent everything that we need
  - Ergonomics: Easy to annotate, no "boilerplate" (thanks to reasoning)

- Motivate the need for annotations vs plain text:
  - Some things cannot be inferred w/o domain knowledge
  - NLP state-of-the-art can't accurately annotate text like we want to, this
  intermediate step helps add structure, and also presents a format to create
  future datasets.

\section{ATD Semantics}

- Describe basic building blocks: Annotations, Relations
  
\subsection{Annotation Types}

- Action
  - Optional
  - ...
 
- Role
  - Person
  - Computer System?

- Object
  - Data Store
  - ...

- Condition / Event


\subsection{Relation Types}

- Agent, Patient

- Coreferences

- Control Flow

- ...

\subsection{Automatic Reasoning}

- Inferring order relationships / transitive closure

- More complex problems (e.g. soundness?)


